%-------------------------
% Resume in Latex
% Author : Morgan Kryze
% Based off of: https://github.com/jakegut/resume
% License : MIT
%------------------------

\documentclass[letterpaper,11pt]{article}

\usepackage{latexsym}
\usepackage[empty]{fullpage}
\usepackage{titlesec}
\usepackage{marvosym}
\usepackage[usenames,dvipsnames]{color}
\usepackage{verbatim}
\usepackage{enumitem}
\usepackage[hidelinks]{hyperref}
\usepackage{bookmark}
\usepackage{fancyhdr}
\usepackage[english]{babel}
\usepackage{tabularx}
\input{glyphtounicode}


%----------FONT OPTIONS----------
% sans-serif
% \usepackage[sfdefault]{FiraSans}
% \usepackage[sfdefault]{roboto}
% \usepackage[sfdefault]{noto-sans}
% \usepackage[default]{sourcesanspro}

% serif
% \usepackage{CormorantGaramond}
% \usepackage{charter}

\setlength{\footskip}{5pt}
\pagestyle{fancy}
\fancyhf{} % clear all header and footer fields
\fancyfoot{}
\renewcommand{\headrulewidth}{0pt}
\renewcommand{\footrulewidth}{0pt}

% Adjust margins
\addtolength{\oddsidemargin}{-0.5in}
\addtolength{\evensidemargin}{-0.5in}
\addtolength{\textwidth}{1in}
\addtolength{\topmargin}{-.5in}
\addtolength{\textheight}{1.0in}

\urlstyle{same}

\raggedbottom
\raggedright
\setlength{\tabcolsep}{0in}

% Sections formatting
\titleformat{\section}{
  \vspace{-4pt}\scshape\raggedright\large
}{}{0em}{}[\color{black}\titlerule \vspace{-5pt}]

% Ensure that generate pdf is machine readable/ATS parsable
\pdfgentounicode=1

%-------------------------
% Custom commands
\newcommand{\resumeItem}[1]{
  \item\small{
    {#1 \vspace{-2pt}}
  }
}

\newcommand{\resumeSubheading}[4]{
  \vspace{-2pt}\item
    \begin{tabular*}{0.97\textwidth}[t]{l@{\extracolsep{\fill}}r}
      \textbf{#1} & #2 \\
      \textit{\small#3} & \textit{\small #4} \\
    \end{tabular*}\vspace{-7pt}
}

\newcommand{\resumeSubSubheading}[2]{
    \item
    \begin{tabular*}{0.97\textwidth}{l@{\extracolsep{\fill}}r}
      \textit{\small#1} & \textit{\small #2} \\
    \end{tabular*}\vspace{-7pt}
}

\newcommand{\resumeProjectHeading}[2]{
    \item
    \begin{tabular*}{0.97\textwidth}{l@{\extracolsep{\fill}}r}
      \small#1 & #2 \\
    \end{tabular*}\vspace{-7pt}
}

\newcommand{\resumeSubItem}[1]{\resumeItem{#1}\vspace{-4pt}}

\renewcommand\labelitemii{$\vcenter{\hbox{\tiny$\bullet$}}$}

\newcommand{\resumeSubHeadingListStart}{\begin{itemize}[leftmargin=0.15in, label={}]}
\newcommand{\resumeSubHeadingListEnd}{\end{itemize}}
\newcommand{\resumeItemListStart}{\begin{itemize}}
\newcommand{\resumeItemListEnd}{\end{itemize}\vspace{-5pt}}

%-------------------------------------------
%%%%%%  RESUME STARTS HERE  %%%%%%%%%%%%%%%%%%%%%%%%%%%%


\begin{document}

%----------HEADING----------
% \begin{tabular*}{\textwidth}{l@{\extracolsep{\fill}}r}
%   \textbf{\href{http://sourabhbajaj.com/}{\Large Sourabh Bajaj}} & Email : \href{mailto:sourabh@sourabhbajaj.com}{sourabh@sourabhbajaj.com}\\
%   \href{http://sourabhbajaj.com/}{http://www.sourabhbajaj.com} & Mobile : +1-123-456-7890 \\
% \end{tabular*}

\begin{center}
  \textbf{\Huge Yann Vidamment} \\ \vspace{1pt}
  \href{mailto:yann.vidamment@edu.devinci.fr}{\underline{yann.vidamment@edu.devinci.fr}} $|$ 
  \href{https://linkedin.com/in/yann-vidamment-80a512254/}{\underline{linkedin.com/in/yann-vidamment}} $|$
  \href{https://github.com/MorganKryze}{\underline{github.com/MorganKryze}}
\end{center}


%-----------EDUCATION-----------
\section{Formation}
  \resumeSubHeadingListStart
    \resumeSubheading
      {École Supérieure d'Ingénieurs Léonard de Vinci}{La Défense, Paris}
      {Ingénieur généraliste, majeure Cloud Computing \& Cybersécurité}{Sept. 2021 -- Aujourd'hui}
    \resumeSubheading
      {Lycée Sainte Jeanne Élisabeth}{Paris}
      {Série générale, Section Européenne anglais, Maths expertes}{Sept. 2018 -- Juin 2021}
  \resumeSubHeadingListEnd


%-----------EXPERIENCE-----------
\section{Expériences}
  \resumeSubHeadingListStart
    \resumeSubheading
      {Président de l'association DeVinci Fablab}{Avril 2023 -- Aujourd'hui}
      {École Supérieure d'Ingénieurs Léonard de Vinci}{La Défense, Paris}
      \resumeItemListStart
        \resumeItem{Pérennisation des activités de l'association, de ces processus et méthodes de gestion (outils, status de l'association)}
        \resumeItem{Suivi des projets : académiques, projets d'étudiants et professionnels avec l'association Junior Entreprise de l'École}
        \resumeItem{Suivi des formations : formats, recrutement de formateurs au sein de l'association et à l'extérieur}
        \resumeItem{Suivi de la trésorerie : négociations des budgets, développement de nos relations avec fournisseurs}
        \resumeItem{Réalisation de prestation de découpe pour l'animalerie de l'université Paris-Saclay}
      \resumeItemListEnd

    \resumeSubheading
      {FabManager au Laboratoire Big Data et IA}{Juin 2023 -- Sept. 2023}
      {ministère des Armées}{Invalides, Paris}
      \resumeItemListStart
        \resumeItem{Création d'un Fablab au sein du ministère des Armées au sein de la DTPM, service transverse chargé des performances ministérielles}
        \resumeItem{Commande de matériels, installation, mise en place de l'espace et initiation aux équipements}
        \resumeItem{Présentation de l'espace sous différents formats : conférences, visites, etc.}
        \resumeItem{Développement d'une application polyvalente open-source citée en section projets}
      \resumeItemListEnd

    \resumeSubheading
      {Responsable projets de l'association DeVinci Fablab}{Avril 2022 -- Mars 2023}
      {École Supérieure d'Ingénieurs Léonard de Vinci}{La Défense, Paris}
      \resumeItemListStart
        \resumeItem{Définition de cahiers des charges pour les projets sur l'année 2022-2023}
        \resumeItem{Recrutements, sélections et suivi des projets tout au long de l'année}
      \resumeItemListEnd
  \resumeSubHeadingListEnd


%-----------PROJECTS-----------
\section{Projets}
    \resumeSubHeadingListStart
      \resumeProjectHeading
          {\href{https://github.com/THE-TRAVELERS/TRAVELERS-HUB}{\textbf{TRAVELERS}} $|$ \emph{Python, FastAPI, git, Dart, Flutter}}{Sept. 2022 -- Aujourd'hui}
          \resumeItemListStart
            \resumeItem{Création d'un rover lunaire tout terrain dans le cadre d'un projet associatif de 6 à 8 étudiants}
            \resumeItem{Contrôle du système électronique des moteurs, drivers, capteurs et télécommunications}
            \resumeItem{Développement d'une application de contrôle et de visualisation des données du rover}
          \resumeItemListEnd
      \resumeProjectHeading
          {\href{https://github.com/KodeLab-fr/ExpenseTracker}{\textbf{Assistant FabLab}} $|$ \emph{Dart, Python, SQL, FastAPI, git, Docker}}{Juin 2023 -- Aujourd'hui}
          \resumeItemListStart
            \resumeItem{Création avec Thomas DERUDDER d'un pilote d'application cross-plateform et polyvalente pour FabLab}
            \resumeItem{Gestion d'utilisateurs avec chiffrage des données}
            \resumeItem{Gestion de stocks, état de machines, événements, formulaires, etc.}
          \resumeItemListEnd
      \resumeProjectHeading
          {\href{https://github.com/MorganKryze/ConsoleAppVisuals}{\textbf{ConsoleAppVisuals}} $|$ \emph{C\#, Bash, git, NuGet, GitHub Packages}}{Sept. 2023 -- Nov.2023}
          \resumeItemListStart
            \resumeItem{Création de librairie C\# facile d'utilisation pour afficher des fonctionnalités graphiques dans la console (sélecteurs, barres de progression, etc.)}
            \resumeItem{Automatisation du déploiement de la documentation  et des tests à l'aide de GitHub Actions et DocFX}
            \resumeItem{Publication de la librairie sur GitHub Packages et NuGet avec plus de 2k téléchargements}
          \resumeItemListEnd
    \resumeSubHeadingListEnd

%-----------PROGRAMMING SKILLS-----------
\section{Compétences Techniques}
 \begin{itemize}[leftmargin=0.15in, label={}]
    \small{\item{
     \textbf{Langues}{: Français (langue maternelle), Anglais courant B2-C1 (TOEFL : 587, TOEIC : 890)} \\
     \textbf{Langages de programmation}{: C\#, C++, Dart, Python, Bash, SQL} \\
     \textbf{Frameworks}{: Flutter (GetX), FastAPI, Arduino} \\
     \textbf{Bibliothèques}{: pandas, NumPy, FastAPI, get, dartz, ConsoleAppVisuals} \\
     \textbf{Outils de développement}{: GitHub, DockerHub, VS Code, Visual Studio, Arduino IDE, XCode, UnityHub} \\
     \textbf{Logiciels généraux}{: Notion, Arc, Fusion360, KiCad, Thunderbird, UTM, Suite Office}
    }}
 \end{itemize}


%-------------------------------------------
\end{document}
